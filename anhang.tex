\section{Einleitung}

Das Nutzungsverhalten des Internets hat sich in den letzten Jahren sehr gewandelt. Noch vor f�nf Jahren bestand die Nutzung gro�teils aus dem Betrachten einzelner Webseiten, die als Dokumente wahrgenommen wurden. Besucht man heute eine der modernen Webseiten, so kann man schon lange nicht mehr von einem Dokument sprechen. Vielmehr sind Webseiten zu Webapplikationen geworden. Anwendungen deren Betriebsystem der Browser ist.
Den gr��ten Schritt in diese Richtung geht Google mit \emph{Chrome OS}, einem Betriebssystem das nur noch aus einem Browser besteht. Alle Applikationen laufen ausschlie�lich im als Webanwendungen im Browser. Die Verf�gbarkeit verschiedener Webapplikationen in allen Bereichen der Computernutzung hat enorm zugenommen und es gibt kaum mehr Aufgaben f�r die ausschlie�lich native Anwendungen zur Verf�gung stehen. So scheint Googles Ziel den Browser als neues Betriebssystem zu etablieren nicht in all zu weiter Ferne.

Eine Reihe an Voraussetzungen muss erf�llt werden, damit der Benutzer von Webanwendungen im Vergleich zur Nutzung nativer Applikationen keine Einbu�en hinnehmen muss: Zun�chst muss die Geschwindigkeit der Internetverbindung ausreiched hoch sein, um die Applikation und deren Inhalte ohne lange Wartezeiten zu �bertragen. Anders als bei nativen Applikationen muss die Applikation selbst bei jeder Nutzung mit�bertragen werden. Obwohl der Browser eine zus�tzliche Schicht zwischen Applikation und Prozessor darstellt, darf die Ausf�hrung nicht wesentlich langsamer sein, als die nativer Applikationen. Sowohl f�r die Geschwindigkeit der Internetverbindungen, als auch f�r die Geschwindigkeit der Ausf�hrung ist bereits heute ein Niveau erreicht, dass die Nutzung von Webapplikationen erm�glicht und f�r die n�chsten Jahre ist noch ein deutliches Wachstum in beiden Bereichen zu erwarten.

Damit aus dem Browser eine Umgebung zur Ausf�hrung von Applikationen wird, m�ssen die Hardwarekomponenten des Computers aus dem Browser heraus verf�gbar sein. Diese Schnittstellen (\emph{APIs}) finden unter dem Schlagwort \emph{HTML5} Einzug in moderne Browser und erm�glichen so beispielsweise den Zugriff auf das Adressbuch des Nutzers, seinen aktuellen Standort, Kamera und Mikrofon oder auch Systemkomponenten wie die Grafikkarte oder den Speicher. All das sind Funktionen, die native Applikationen schon immer nutzen konnten, f�r Webanwendungen aber erst seit kurzer Zeit zur Verf�gung stehen. Das stellt die Browserhersteller mitunter auch vor Sicherheitsprobleme, da diese Zugriffe keinesfalls unauthorisiert erfolgen d�rfen. 

Vorteilhaft an Webapplikationen ist unter anderem die schnelle und einfache Verf�gbarkeit. Ohne einlegen eines Datentr�gers und ohne Installationsprozess steht die Anwendung sofort zur Verf�gung. Da die Daten bereits online gespeichert sind ist Kollaboration und Synchronisation zwischen mehreren Rechnern und Benutzern einfacher m�glich.

F�r den Applikations-Entwickler stellt der Browser eine M�glichkeit zur plattform�bergreifenden Entwicklung dar. Durch Allianzen der Browserhersteller und die Festsetzung gemeinsamer Standards ist es, abgesehen von kleinen Anpassungen, meist m�glich eine Webapplikation unter beliebigen Betriebsystemen und Browsern auszuf�hren. In den Standards definierte Technologien erfodren �blicherweise auch keine Installation eines Plug-In. Lediglich die unterschiedliche Verf�gbarkeit verschiedener Technologien ist bisher noch ein Problem, das der Entwickler bedenken muss.

Die Entwicklung und der Einsatz von Webapplikationen wird in den kommenden Jahren weiter zunehmen und viele native Anwendungen vom Markt verdr�ngen. Trotzdem wird es sicherlich weiterhin Bereiche geben, in denen der Einsatz einer nativen Anwendung sinnvoller oder notwendig ist.


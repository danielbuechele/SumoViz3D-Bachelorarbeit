\chapter{Dreidimensionale Computergrafik im Browser}

\section{Historische Entwicklung der 3D-Technologie im Web}
\label{hist}

Erste Bem�hungen dreidimensionale Inhalte in den Webbrowser zu bringen begannen im Jahr 1994. Angelehnt an die Nutzung von HTML f�r die Darstellung von Webseiten sollte eine Auszeichnungssprache f�r die Darstellung dreidimensionaler Inhalte geschaffen werden. Anforderungen an die neue Sprache waren Plattformunabh�ngigkeit, Erweiterbarkeit und Nutzbarkeit auch bei geringer Internet-Bandbreite. Das zu diesem Zweck gegr�ndete \emph{Web3D}-Konsortium entwickelte die \emph{Virtual Reality Modeling Language (VRML)}. Am 26. Mai 1995 wurde die finale Spezifikation von \emph{VRML 1.0} ver�ffentlicht. Dort wird ein einfaches textbasiertes Format definiert, in dem Objekte geschachtelt abgelegt werden k�nnen. Diese Objekte k�nnen Kameras, Lichter, Materialien, aber auch dargestellte 3D-Objekte oder Transformationen sein. Zur Speicherung der Szenen wird die Dateiendung \verb+.wrl+ (f�r engl. \emph{world}) vorgegeben. \cite{vrml10} 1997 wurde \emph{VRML 2.0 (VRML97)} fertiggestellt und als ISO-Standard 14772-1:1997 definiert. Unter anderem wurde der VRML-1.0-Standard um Animationen und Nutzerinteraktionsm�glichkeiten erweitert. \cite{vrml97} \cite{vrml972}

F�r die Darstellung von VRML-Inhalten gibt es einige Browser-Plugins, jedoch integrierete nahezu kein Browser-Hersteller den Standard direkt in sein Produkt. In den folgenden Jahren entstanden viele weitere Dateiformate zur Speicherung von 3D-Szenen mit Augenmerk auf die Darstellung im Browser. Unter anderem trat das XML-basierte Format \emph{X3D} die Nachfolge von VRML an. Das Web3D-Konsortium schl�gt vor den X3D-Standard komplett in HTML5 zu integrieren, jedoch hat das W3C das bisher abgelehnt: \emph{``Embedding 3D imagery into XHTML documents is the domain of X3D, or technologies based on X3D that are namespace-aware.''} \cite{nox3d} �hnlich wie bei \emph{SVG} oder \emph{MathML} soll so eine native Darstellung ohne Plugin erm�glicht werden. \cite{x3domhtml}



\section{Deklarative und imperative 3D-Darstellung in Browsern}


Die Darstellung von zwei- und dreidimensionalen Grafiken im Browser kann grundlegend auf zwei verschiedene Arten erfolgen: Zum einen ist es m�glich den Szenengraph als Teil des HTML-Dokuments zu deklarieren. Somit sind die einzelnen Objekte der Grafik gleichberechtigt mit HTML-Elementen auf der Seite. Jedes Objekt der Grafik ist im DOM-Baum integriert und kann per CSS oder JavaScript selektiert und modifiziert werden. Zur Darstellung zweidimensionaler Grafiken ist das \emph{SVG}-Format im HTML5-Standard aufgenommen. Das �quivalent f�r dreidimensionale Grafiken, X3D, ist wie in Kapitel \ref{hist} beschreiben nicht Teil des Standards. \cite{x3dom} 

Zum anderen k�nnen Grafiken imperativ auf eine Zeichenfl�che gezeichnet werden. Als Zeichenfl�che wird das canvas-Element genutzt und ist dabei das einzige Objekt der Grafik, das im DOM-Baum verankert ist. Inhalte die auf die Zeichenfl�che gezeichnet wurden k�nnen nicht mehr ver�ndert werden. Die Zeichenfl�che kann lediglich komplett oder in Ausschnitten geleert und/oder �berzeichnet werden. HTML5 definiert die 2D-F�higkeiten des canvas-Elements und Verwendung von WebGL f�r 3D-Inhalte. \cite{2dcontext}

\begin{table}[h]

\centering
\begin{tabular}{l||c|c}
     & \textbf{2D} & \textbf{3D} \\
\hline
\hline
    \textbf{Deklarativ} & SVG & X3D\footnotemark \\
\hline
    \textbf{Imperativ} & canvas & WebGL
\end{tabular}

\label{tbl:tablelabel}
\caption{Deklarative und imperative Grafikdarstellung im Web \cite{x3dom}}
\end{table}
\footnotetext{nicht vom W3C standardisiert}

%http://www.w3.org/TR/2dcontext/



\subsection{Importieren der Ausgangsdaten}

Die Ergebnisse der Simulation werden in eine Text-Datei geschrieben die zur Verwendung mit SumoViz eingelesen und in einer Datenbank abgelegt wird. Die Datens�tze sind zeilenweise in der Ausgangsdatei abgelegt und in Bl�cke unterteilt. Bl�cke innerhalb der Datei werden mit \verb+Begin+ und \verb+End+ sowie dem Blocknamen gekennzeichnet. Beispielsweise sind die Daten der Baugeometrie im Block \verb+Begin geometry+ angelegt. Die Ausgangsdatei enth�lt immer einen Block mit Parametern zur Simulation, die in SumoViz nicht verwendet werden, da die Parameter haupts�chlich f�r die Erstellung der Simulation, nicht aber f�r die Visualisierung n�tig sind.


Zum Importieren der Daten in die CouchDB-Datenbank exsistiert das node.js-Skript dataimport.js, das die Datei block- und zeilenweise Verarbeitet, die Daten in






Durch Zeilen mit den Schl�sselw�rtern \verb+Begin Data+ und \verb+End Data+ ist innerhalb der Ausgabedatei ein Abschnitt mit den Fu�g�ngerpositionsdaten ausgezeichnet. Die Datens�tze werden zeilenweise angegeben und beinhalten pro Datensatz einen Zeitstempel, eine Identifikationsnummer des Fu�g�ngers, Positionsdaten und Angaben zur Ebene auf der sich der Fu�g�nger befindet sowie zur Dichte. Pro Zeitpunkt und sichtbaren Fu�g�nger wird eine Zeile in der Ausgabedatei generiert. Die Werte des Datensatzes sind mit Leerzeichen getrennt.
\begin{table}[h]
\centering
\begin{tabular}{l||l|l|l|l|l|l|l}
& \textbf{timecode} & \textbf{pedid} & \textbf{x} & \textbf{y} & \textbf{z} & \textbf{level} & \textbf{density} \\
\hline
\hline
\textbf{Datentyp} & \verb+float+ & \verb+int+ & \verb+float+ & \verb+float+ & \verb+float+ & \verb+int+ & \verb+float+ \\
\hline
\textbf{Beispiel} & \verb+14.02+ & \verb+32+ & \verb+30.06+ & \verb+36.04+ & \verb+0.00+ & \verb+0+ & \verb+0.17+ \\
\end{tabular}
\label{tbl:tablelabel}
\caption{Fu�g�ngerdatenformat}
\end{table}
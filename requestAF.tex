\subsection{requestAnimationFrame}

Um Animationen in JavaScript auszuf�hren wurden die Funktionen \verb+setTimeout(function,delay)+ oder \verb+setInterval(function,interval)+ verwendet. Diese rufen dann in einem festgelegten Rhythmus eine Funktion auf, die inkrementell die Animation ausf�hrt. W�hrend \verb+setInterval+ nur einmalig aufgerufen wird und dann von alleine die Funktion im angegebenen Intervall aufruft, muss \verb+setTimeout+ nach der Ausf�hrung des Animationsschritts erneut rekursiv aufgerufen werden. Jedoch gibt es einige Nachteile mit der Verwendung dieser Methoden: Die Wiederholrate wird fest vorgegeben, unabh�ngig von der Leistung des Rechners. Ist die Ausf�hrung zum angegbenen Zeitpunkt nicht m�glich, da gerade eine andere Berechnung ausgef�hrt wird, wird die Funktion in eine Warteschlange gestellt. In der Warteschlange wird nur ein Aufruf jeder Funktion gespeichert, alle weiteren werden ignoriert und nicht ausgef�hrt. �blicherweise liegt die Wiederholrate eines Bildschirms bei $60Hz$. Daher lohnt es sich nicht die Animations-Schleife �fter auszuf�hren. Damit ergibt sich eine Intervallzeit von ca. $17ms$.

%http://www.nczonline.net/blog/2011/05/03/better-javascript-animations-with-requestanimationframe/
%http://ejohn.org/blog/how-javascript-timers-work/


Ein grundlegendes Problem bei der beschriebenen Vorgehensweise ist, das f�r den Browser nicht erkennbar ist, das es sich bei der aufgerufenen Funktion um die Ausf�hrung einer Animation handelt und deshalb keine spezifischen Optimierungen m�glich sind. Mozilla entwickelte den Funktionsaufruf \verb+requestAnimationFrame+. Da der Browser wei�, dass es sich dabei um die Ausf�hrung von Animationen hadelt kann er beispielsweise parallel ablaufende Aufrufe in einem repaint-Zyklus zusammenfassen. Wie bei der \verb+setTimeout+-Methode muss die Funktion nach dem ablaufen des Animationsschritts rekursiv aufgerufen werden, wodurch die Animation mit maximaler Wiederholrate abl�uft. Der Browser kann die Ausf�hrung komplett stoppen, wenn das Fenster im Hintergrund ist und nicht f�r den User sichtbar. Das spart CPU-Ressourcen und senkt so den Stromverbrauch.

%http://paulirish.com/2011/requestanimationframe-for-smart-animating/
%https://developer.mozilla.org/en-US/docs/DOM/window.requestAnimationFrame

Die Standardisierung vom W3C ist noch nicht final, weshalb einige Browser-Hersteller die Funktion mit einem so genannten Vendor-Prefix implementieren. F�r die Verwendung in Firefox wird beispielsweise \verb+mozRequestAnimationFrame+ verwendet. Deshalb setzt SumoViz3D ein \emph{Polyfill} ein, das beim Aufruf von \verb+requestAnimationFrame+ die im jeweiligen Browser verf�gbare Methode verwendet. F�r Browser, ohne Unterst�tzung von \verb+requestAnimationFrame+ gibt ein Fallback mit der Verwendung von \verb+setTimeout+, das �ber  die Intervallzeit anzupassen.

%http://www.w3.org/TR/animation-timing/#requestAnimationFrame

Die Wiederholrate der Animation ist abh�ngig von der Geschwindigkeit des Rechners. Der Ablauf der Visualisierung muss aber unabh�ngig von der Rechneleistung sein, damit auf schnellen Rechnern nicht alle Simulationsschritte innerhalb weniger Sekunden ablaufen. Deshalb wird das Aktualisieren der Fu�g�ngerkoordinaten und deren Einf�rbung von der Animationsschleife entkoppelt und mit einer \verb+setInterval+-Methode regelm��ig ausgef�hrt. Innerhalb der Animationsschleife findet dann das Neuzeichnen des canvas-Elements und das Aktualisieren der Kameraposition (vgl. Kapitel \ref{controls}) statt. Zus�tzlich wird noch das THREE.js-Stats-Objekt zum Auslesen der Framerate aktualisiert.

\begin{verbatim}
setInterval(updatePedestrians, 1000/5);  //
animate();                               //Initialer Aufruf der Animationsschleife

function animate() {
    requestAnimationFrame( animate );    //rekursiver Aufruf der Animationsschleife
    controls.update();                   //Aktualisieren der Kameraposition
    renderer.render( scene, camera );    //Neuzeichnen des canvas-Elements
    stats.update();                      //Messen der Geschwindigkeit
}
\end{verbatim}
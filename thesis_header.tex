% !TEX root =  thesis.tex
\documentclass[11pt,oneside,a4paper]{book}
\usepackage[latin1]{inputenc}
\usepackage{amsmath}
\usepackage{amssymb}
%f�r das Einbinden von Bildern und Tabellen
\usepackage{booktabs} 		%f�r tolle tabellen
\usepackage{threeparttable}
\usepackage{longtable}
%\usepackage[normal,small,bf]{caption} %erm�glicht mehrzeilige Bildunterschriften use [hang] f�r einger�ckte Bildunterschrift


\usepackage{psfrag}
%f�r die Bibliography
\usepackage{natbib}		% more flexibility with citations
\usepackage{ifpdf}
\usepackage{footnote}
%Verwendung von Farben
\usepackage{xcolor}
%bedruckten Bereich der Seite anpassen
\usepackage[left=3.0cm, right=2.5cm, top=2.5cm,bottom=2.0cm,includeheadfoot]{geometry}
% Formatierung Kapitelueberschriften
\usepackage{titlesec}
\usepackage[T1]{fontenc} % sonst geht \hyphenation nicht mit Umlauten
\usepackage{listings}
\usepackage{color}
\usepackage{wrapfig}
\usepackage{tikz}
\newcommand*\circled[1]{\tikz[baseline=(char.base)]{
            \node[shape=circle,draw,inner sep=1pt] (char) {#1};}}
 
\usepackage{lmodern}

\lstset{ %
  language=Java,                % the language of the code
  basicstyle=\ttfamily\footnotesize,           % the size of the fonts that are used for the code
  numbers=left,                   % where to put the line-numbers
  numberstyle=\tiny\color{gray},  % the style that is used for the line-numbers
  stepnumber=2,                   % the step between two line-numbers. If it's 1, each line 
                                  % will be numbered
  numbersep=5pt,                  % how far the line-numbers are from the code
  backgroundcolor=\color{white},      % choose the background color. You must add \usepackage{color}
  showspaces=false,               % show spaces adding particular underscores
  showstringspaces=false,         % underline spaces within strings
  showtabs=false,                 % show tabs within strings adding particular underscores
  frame=shadowbox,                   % adds a frame around the code
  rulecolor=\color{gray},        % if not set, the frame-color may be changed on line-breaks within not-black text (e.g. commens (green here))
  tabsize=2,                      % sets default tabsize to 2 spaces
  captionpos=b,                   % sets the caption-position to bottom
  breaklines=true,                % sets automatic line breaking
  breakatwhitespace=false,        % sets if automatic breaks should only happen at whitespace
  title=\lstname,                   % show the filename of files included with \lstinputlisting;
                                  % also try caption instead of title
  keywordstyle=\color{black},          % keyword style
  commentstyle=\color{black},       % comment style
  stringstyle=\color{black},         % string literal style
  escapeinside={\%*}{*)},            % if you want to add LaTeX within your code
  morekeywords={*,...}               % if you want to add more keywords to the set
}



%\titleformat{?�berschriftenklasse?}[Absatzformatierung?]{?Textformatierung?} {?Nummerierung?}{?Abstand zwischen Nummerierung und �berschriftentext?}{?Code vor der �berschrift?}[?Code nach der �berschrift?]
%\titleformat{\chapter}[hang]{\huge\bfseries}{\thechapter\quad}{0pt}{}
%\titlespacing{?�berschriftenklasse?}{?Linker Einzug?}{?Platz oberhalb?}{?Platz unterhalb?}[?rechter Einzug?]
%\titlespacing{\chapter}{0pt}{0em}{6pt}
%----------------------------------------------------------------------------------------------------------------------------------------------
%Stil der Seite
\usepackage{fancyhdr}
\fancypagestyle{plain}{%
	\fancyhf{} % clear all header and footer fields
	\fancyhead[EL,OR]{\slshape \thepage} % except the center
	\fancyhead[ER]{\slshape \leftmark}
	\fancyhead[OL]{\slshape \rightmark}
}
\renewcommand{\chaptermark}[1]{\markboth{\thechapter. #1}{}}
\renewcommand{\sectionmark}[1]{\markright{\thesection. #1}{}}
%
\pagestyle{empty}%f�r die Titelseiten zumindest, am Ende der Titelseite umschalten auf \pagestyle{fancyplain}
\pagenumbering{Roman}%f�r die Titelseiten zumindest, am Ende der Titelseite umschalten auf \pagenumbering{arabic}
%------------------------je nach Kommando (pdflatex / latex) jeweilige Paketeinbindung-----------------------------
\definecolor{linkblue}{rgb}{0,0.1,0.6}
\definecolor{citegreen}{rgb}{0,0.25,0.15}%{0.1,0.5,0.4}%{0.125,0.6,0.5}
\definecolor{linkred}{rgb}{0.8,0,0.005}%{0.6,0,0.1}
\definecolor{mailviolet}{rgb}{0.3,0,0.35}%{0.6,0,0.1}
\definecolor{tumblue}{rgb}{0,0.396,0.741}

% Zum Setzen von URLs
\definecolor{darkred}{rgb}{.25,0,0}
\definecolor{darkgreen}{rgb}{0,.2,0}
\definecolor{darkmagenta}{rgb}{.2,0,.2}
\definecolor{darkcyan}{rgb}{0,.15,.15}
\usepackage[plainpages=false,bookmarks=true,bookmarksopen=true,colorlinks=true,
  linkcolor=darkred,citecolor=darkgreen,filecolor=darkmagenta,
  menucolor=darkred,urlcolor=darkcyan]{hyperref}


% pdflatex: Bilder in den Formaten .jpeg, .png und .pdf
% latex: Bilder im .eps-Format
\usepackage{graphicx}

\usepackage[ngerman]{babel}
\usepackage[latin1]{inputenc}
\usepackage[T1]{fontenc}

%Zeilenabst�nde regulieren mit \singlespacing, \onehalfspacing, \doublespacing
\usepackage{subfig} %{subfigure} alt
\usepackage{setspace}
\onehalfspacing
\parindent 0pt
%Schriftart Sans Serif
%\renewcommand*\familydefault{\sfdefault} 




